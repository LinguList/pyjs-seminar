\documentclass[xetex,svgnames]{scrartcl}

% packages
\usepackage{xltxtra}
\usepackage{polyglossia}
\usepackage{lsalike}
\usepackage{hyperref}
\usepackage{fontspec}
\usepackage{covington}
\usepackage{scrpage2}
\usepackage{qtree}
\usepackage[left=2cm,right=2cm,top=3cm,bottom=3cm]{geometry}
\usepackage{draftwatermark}
\usepackage{hieroglf}
%\usepackage{simpsons}
\usepackage[weather,misc,alpine]{ifsym}
\usepackage{phaistos}
\usepackage{linearb}
\usepackage{dingbat}
\usepackage{pst-node}
\usepackage{colortbl}
\usepackage{alltt}
\usepackage{listings}

% fonts general
\setmainfont[Mapping=tex-text,Scale=1.0]{FreeSans}
\setsansfont[Mapping=tex-text,Scale=1.0]{FreeSans}
\setmonofont{FreeMono}

% special fonts
\newfontfamily\hana{HAN NOM A}
\newfontfamily\hanb{HAN NOM B}
\newfontfamily\sil{Doulos SIL}
\newfontfamily\grk{Aristarcoj}
\newfontfamily\calligraphy{Chinese Calligraphy}
\newfontfamily\gnm{Gnommish}
\newfontfamily\jin{cjkbronze}
\newfontfamily\jiagu{cjkjiagu}
\newfontfamily\xiaozhuan{shuowenxiaozhuan}
\newfontfamily\pur{Purisa}
\newfontfamily\cjcalligraphy{sinocalligraphy}
%\newfontfamily\ahd{eufm10}

% specific commands
\newcommand{\mysub}[1]{\raisebox{-0.5ex}{\scriptsize{#1}}}
\SetWatermarkText{PYTHON}
\newcommand{\bild}[2]{%
    \scalebox{#1}{%
        \includegraphics{/home/mattis/projects/graphics/img/#2.jpg}
        }
    }
\newcommand{\Table}[1]{%
    \begin{flushleft}
        \begin{tabular}{|p{16.5cm}|}
            \hline \cellcolor{lightgray} \bf \pur #1
            \\\hline
        \end{tabular}
    \end{flushleft}
}

\lstset{language=Python}

\newcommand{\Code}[1]{%
    \begin{flushleft}
        \begin{tabular}{||p{16.5cm}||}
            \hline\hline \cellcolor{white}
            \\
            #1

             \cellcolor{white}
            \\\hline\hline
        \end{tabular}
    \end{flushleft}
}
\newcommand{\White}[1]{\cellcolor{white} \textcolor{black}{ #1}}

% language settings
\setmainlanguage[spelling=new]{german}
\setotherlanguage{english}

% pagestyle settings
\pagestyle{scrheadings}
\ihead{Johann-Mattis List}
\chead{Python und JavaScript}
\ohead{Übungen 1}
\ifoot{}
\cfoot{\pagemark}
\ofoot{}

%\include{latex/figures}

\begin{document}
%\maketitle

\begin{center}
  \bf \huge Praktisches zu Python
\end{center}

\Table{Um dem Seminar optimal folgen zu können, sollten Sie sicherstellen, dass Sie alle wichtigen
Programme installiert haben. In Bezug auf Python wären das:
\begin{itemize}
  \item Python selbst (\url{http://python.org}), und zwar in der Version 3.4.,
  \item NumPy (\url{http://numpy.org}),
  \item SciPy (\url{http://scipy.org}),
  \item Matplotlib (\url{http://matplotlib.org}),
  \item Networkx (\url{http://networkx.org}),
  \item LingPy (\url{http.lingpy.org})
\end{itemize}

Damit Sie überprüfen können, ob Sie auch alle Installationen richtig ausgeführt haben, habe ich ein kleines,
sehr einfaches Skript geschrieben. Wenn Sie dieses ausführen, dann gibt es am Ende einen Lösungssatz
aus. Wie lautet dieser Satz?}
\begin{center}
  \bf \huge Praktisches zu JavaScript
\end{center}
\Table{Um gut in JavaScript einsteigen zu können, ist es zunächst am wichtigsten, dass Sie die Tools haben,
die Sie für ordentliche Webprogrammierung brauchen. Aufgrund der Unübersichtlichkeit des Codes ist
es vor allem auch für JavaScript unerlässlich, ein Versionsverwaltungssystem zu verwenden. Um Ihre
Daten zu sichern und mit Kollegen zu teilen, sollten Sie ferner die Dienste eines Webhosting
Services in Anspruch nehmen. Mit GitHub (\url{http://github.com}) ist dies am einfachsten. 
\\
\bf\pur\cellcolor{lightgray}
Beginnen Sie also ihren ersten Schritt in Richtung von JavaScript damit, dass Sie sich einen GitHub
account anlegen, falls Sie den noch nicht haben. 
Dann erstellen Sie bitte als nächstes einen Fork des diesem Seminar zugrunde liegenden GitHub
Repositories (\url{http://github.com/lingulist/pyjs-seminar/}), und stellen Sie das Seminar auch
unter ``Beobachtung" (Button ``watch"). Sobald Sie das getan haben, ist es mir möglich, Sie als
Teilnehmer meines Seminars zu identifizieren, und ich sehe, wie viele der Teilnehmer, einen GitHub
Account angelegt haben.
\\
\bf\pur\cellcolor{lightgray}Sobald Sie das getan haben, versuchen Sie doch schon mal, eine erste
HTML/CSS/JS-Applikation zu erstellen. Alles, was Sie dafür brauchen, ist eine Blanko-Datei-Struktur,
wie sie Sie unter \url{https://github.com/LinguList/pyjs-seminar/blob/master/website/code/html-app/}
finden. Schaffen Sie es, mit Hilfe beliebig vieler Internetrecherchen ein kleines Programm zu
entwickeln, durch das die HTML-Website dynamisch bei Knopfdruck auf einen Button um den Satz "Hallo Welt!" ergänzt
wird?
}
\pagebreak

\begin{center}
    {\bf \huge  Jamie Olivers Bratäpfel}
\end{center}

\begin{flushleft}{\bf \Large \textcolor{Crimson}{Bratäpfel}} \\
    \bf \textcolor{Crimson}{Zutaten} \end{flushleft}
    \tabular{ll}
    \em 50g Butter & \em 100g heller Muscovado-Zucker \\
    \em 4 große herbe Kochäpfel & \em 75g Rosinen \\
    \em 2 Lorbeerblätter, getrocknet oder frisch & \em 1 gehäufter TL
    Lebkuchengewürz \\
    \em 2 Gewürznelken & \em 1 Schuss Brandy oder Whiskey \\
    \em 50g Mandelblättchen & \\
    \endtabular
    \begin{flushleft}{\bf \textcolor{Crimson}{Die Äpfel vorbereiten} } \end{flushleft}
\begin{enumerate}
    \item Die Butter 15-30 Minuten, bevor Sie anfangen, aus dem Kühlschrank
        nehmen und etwas weich werden lassen.
    \item Den Backofen auf 180°C vorheizen.
    \item Mit einem Apfelausstecher das Kerngehäuse der Äpfel entfernen, dann
        die Äpfel vorsichtig mit einem Messer in der Mitte rundum einschneiden.
    \item Die Äpfel in eine ofenfeste Form setzen.
    \item Wenn Sie getrocknete Lorbeerblätter verwenden, diese in kleine Stücke
        reißen, frische Blätter fein hacken.
    \item Zusammen mit den Nelken im Mörser zerstoßen.
    \item Mit dem Großteil der Mandeln und den restlichen Zutaten in eine große
        Schüssel geben.
    \item Den Schüsselinhalt mit den Händen gut vermischen und verkneten,
        sodass die Butter alle Aromen aufnehmen kann.
    \item Die Äpfel mit der Buttermasse füllen (da, wo vorher das Kerngehäuse
        war) und mit den Resten der Äpfel außen einreiben.
    \item Die übrigen Mandelblätchen in der verbliebenen Flüssigkeit in der
        Schüssel wenden, dann über die Äpfel streuen.
\end{enumerate}
\begin{flushleft}\bf \textcolor{Crimson}{Die Äpfel braten}\end{flushleft}
    \begin{enumerate}
        \item Die Äpfel 35-40 Minuten im vorgeheizten Ofen braten, bis sie
            goldbraun und weich sind.
        \item Aus dem Ofen nehmen  und vor dem Servieren etwa 5 Minuten
            abkühlen lassen. 
        \item Jeden Apfel in eine kleine Schale setzen und mit dem
            karamellisierten Saft aus der Form beträufeln. 
        \item Sie schmecken fantastisch mit etwas Vanilleeis, Crème fraîche
            oder Vanillesauce.
    \end{enumerate}
%\end{quote}
\Table{
Wir schreiben das Jahr 2030. Die Linguisten sind in der Mitte der Gesellschaft
angekommen, seit die Firma SkyNet den Prototypen eines Hausroboters
herausgebracht hat, welcher auf Befehle in natürlicher Sprache (ausgenommen
Chinesisch, wegen der überhöhten Ambiguitätsrate) reagiert. Sie (ein Absolvent
des berühmten Heinrich Heine Centers for Advanced Linguistics) arbeiten als
Kommunikationstrainer im Bereich MtTUWYM (Make the Thing Understand What You
Mean) für eine große kulinarische Filiale (McDonalds). Ihre Aufgabe ist es,
ein Template für die neue Bratapfelserie (Slogan: ``Das ganze Jahr wie
Weihnachten!) zu entwickeln, welches den letzten lebenden Angestellten hilft,
die Maschinen eindeutig zu trainieren. Ihre Vorlage ist das berühmte
Bratapfelrezept von Jamie Oliver, welches Sie maschinentauglich machen sollen.
Gehen Sie dabei wie folgt vor:
\begin{enumerate}
    \item Kreuzen Sie alle Stellen an, an denen ein Roboter Sie missverstehen
        könnte.
    \item Prüfen Sie exemplarisch am Programmteil 3 "Die Äpfel braten", über
        welches Weltwissen die Maschinen verfügen müssen, um die Aufgabe zu
        bearbeiten.
    \item Ihr Arbeitgeber hat Ihnen aufgetragen, das Rezept als Funktion zu
        programmieren. Lässt sich das bewerkstelligen? Und wenn ja, welche
        Argumente würden sich für die Funktion anbieten?
    \item Übertragen Sie die ersten 5 Sätze des Rezepts in URL (Unambiguous
        Robot Language).
\end{enumerate}
}

%\section*{Literatur}
%\bibliographystyle{lsalike}
%\renewcommand\refname{Literatur}
%\renewcommand{\section}[2]{}
%\bibliography{/home/mattis/Dropbox/EvoClass/bibliography/evoclass}
\end{document}

